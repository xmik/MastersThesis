
\documentclass[12pt]{article}
\usepackage[utf8]{inputenc}
\usepackage{float}
\usepackage[a4paper,hmargin=1.5cm,vmargin=2cm]{geometry}
\usepackage{graphicx}
\usepackage{amsfonts}
\usepackage{textcomp}
\usepackage{hyperref}
\usepackage{listings}
\usepackage{array}
\usepackage{tabularx}
\usepackage{mathtools}
\usepackage{longtable}
\usepackage{enumitem}
\setitemize{noitemsep,topsep=0pt,parsep=0pt,partopsep=0pt}
\setenumerate{noitemsep,topsep=0pt,parsep=0pt,partopsep=0pt}

\usepackage{wrapfig}


\graphicspath{{figures/}}

\title{
{\small Ewa Czechowska } \\
\bf\textit{ Master’s Thesis Description } \\
\vspace{4cm}}
\date{\today}


\begin{document}

\maketitle
~\vspace{8cm}
\newpage
\thispagestyle{empty}

\section{Topic and Supervisor}
\textbf{English topic}: Kubernetes cluster deployment for production environment
\\
\textbf{Polish topic}: Wdrażanie klastra Kubernetes dla produkcyjnego środowiska
\\
\textbf{Supervisor}: dr hab. inż. Aneta Poniszewska-Marańda

\section{Thesis' aims}
\begin{enumerate}
    \item Comparison of chosen methods of Kubernetes cluster deployment. 
    \item Presentation of encountered problems while applying practical approach.
\end{enumerate}


\section{The research problem and the novelty}
\subsection{Problem solutions}
The problem will be solved by:
\begin{enumerate}
    \item (theoretical part) gathering the requirements of a production deployment (e.g. backups, autoscaling, central logs) by analyzing the literature
    \item (theoretical part) describing $>1$ methods of Kubernetes cluster deployment by analyzing the literature
    \item (practical part) comparing $>1$ methods of Kubernetes cluster deployment on the AWS cloud. The criteria could be:
    \begin{itemize}
        \item cost of AWS resources
        \item the amount of problems encountered while applying each approach
        \item the amount of resources that could not be automated
        \item whether the method fulfills the production deployment requirements (e.g. is it customizable enough)
    \end{itemize}
    \item Listing the encountered problems (that occurred while deployment) and providing solutions (if found)
\end{enumerate}


\subsection{The novelty}

The novelty of this work:
\begin{itemize}
    \item attempt to satisfy most of the production deployment requirements (usually the focus is on deploying a 'just good enough' Kubernetes cluster)
    \item the cluster deployment will have automated tests
\end{itemize}

Thanks to the focus on the practical approach and real-life problems, the thesis might help the engineers responsible for Kubernetes cluster deployment plan such deployment better because they would be aware upfront of its limitations and known issues.


\section{Thesis sections}

\begin{lstlisting}
    1. Introduction
    1.1 Topic and work boundaries
    1.2 Aims
    1.3 Research methodology
    1.4 Background and related work
    1.5 Structure of this thesis
    2. Theoretical part: definitions and requirements
    2.1 Rudimentary definitions
    2.1.1 DevOps and microservices
    2.1.2 Production deployment requirements
    2.1.3 Docker Containers
    2.1.4 Kubernetes as Docker Containers Orchestration System
    2.1.5 Kubernetes architecture
    2.1.6 AWS - The Amazon Cloud
    2.2 Kubernetes cluster deployment methods
    2.2.1 Using AWS EKS
    2.2.2 Using kops to deploy on AWS
    2.2.3 Using kops to deploy on GCP 
    2.2.4 ... (other available methods)
    3. Practical part
    3.1 Preparations
    3.1.1 Chosen requirements of production deployment
    3.1.1 Designing automated tests
    3.1.2 Designing infrastructure on AWS
    3.1.3 Defining desired backup strategy
    3.1.4 Defining desired autoscaling strategy
    3.1.5 ... (satisfying other production deployment requirements,
        e.g. planning/designing)
    3.2 Kubernetes cluster deployment using various methods
    3.2.1 Using AWS EKS
    3.2.2 Using Kops to deploy on AWS
    3.3 Comparison of the used methods
    3.3.1 Cost
    3.3.2 The amount of problems encountered while applying each approach
    3.3.3 The amount of resources that could not be automated
    3.3.4 Whether the method fulfills the production deployment requirements
    3.3.5 ... (maybe more criteria)
    3.3 Results
    4. Summary
    4.1 Lessons learned
    4.2 Future work potential    
\end{lstlisting}


\section{Expected results}
\begin{enumerate}
    \item Ability to describe what a production deployment means and to enumerate its requirements.
    \item Ability to describe $>1$ Kubernetes cluster deployment methods.
    \item Successful deployment of a production kubernetes cluster using method A (AWS EKS).
    \item Successful deployment of a production kubernetes cluster using method B (Kops and AWS).
    \item Comparison of the two methods.
    \item Listing of problems encountered while using the two methods. If possible, adding solutions or recommendations how to troubleshoot them.
    \item Built a knowledge base (or put toghether the terms) that must be learned before deploying a Kubernetes cluster.
\end{enumerate}

\newpage

\section{Literature}

Books:
\begin{itemize}
    \item DevOps with Kubernetes \url{https://ebookpoint.pl/ksiazki/devops-with-kubernetes-second-edition-hideto-saito-hui-chuan-chloe-lee-cheng-yang-wu,e_14ou.htm}
    \item Cloud Native DevOps with Kubernetes. Building, Deploying, and Scaling Modern Applications in the Cloud \url{https://ebookpoint.pl/ksiazki/cloud-native-devops-with-kubernetes-building-deploying-and-scaling-modern-applications-in-the-clo-john-arundel-justin-domingus,e_11u7.htm}
    \item Continuous Delivery: Reliable Software Releases through Build, Test, and Deployment Automation (Addison-Wesley Signature Series (Fowler)) \url{https://www.amazon.com/dp/0321601912?tag=contindelive-20}
\end{itemize}
\\
\\
Whitepapers:
\begin{itemize}
    \item AWS Well-Architected Framework \url{https://d1.awsstatic.com/whitepapers/architecture/AWS_Well-Architected_Framework.pdf}
    \item Architecting for the Cloud: Best Practices  \url{https://media.amazonwebservices.com/AWS_Cloud_Best_Practices.pdf}
\end{itemize}
\\
\\
Other sources:
\begin{itemize}
    \item https://12factor.net/
    \item https://gruntwork.io/guides/kubernetes/how-to-deploy-production-grade-kubernetes-cluster-aws
    \item https://platform9.com/blog/kubernetes-cloud-services-comparing-gke-eks-and-aks/
    \item  https://www.presslabs.com/blog/kubernetes-cloud-providers-2019/
\end{itemize}


\end{document}
