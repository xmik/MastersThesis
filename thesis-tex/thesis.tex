
\documentclass[12pt]{article}
\usepackage[utf8]{inputenc}
\usepackage{float}
\usepackage[a4paper,hmargin=1.5cm,vmargin=2cm]{geometry}
\usepackage{graphicx}
\usepackage{amsfonts}
\usepackage{textcomp}
\usepackage{hyperref}
\usepackage{listings}
\usepackage{array}
\usepackage{tabularx}
\usepackage{mathtools}
\usepackage{longtable}
\usepackage{enumitem}
\setitemize{noitemsep,topsep=0pt,parsep=0pt,partopsep=0pt}
\setenumerate{noitemsep,topsep=0pt,parsep=0pt,partopsep=0pt}

\usepackage[backend=bibtex]{biblatex}
\bibliography{thesis-tex/bibliography.bib}


\usepackage{wrapfig}


\graphicspath{{figures/}}

\title{
{\small Ewa Czechowska } \\
\bf\textit{ Master’s Thesis Description } \\
\vspace{4cm}}
\date{\today}


\begin{document}

\maketitle
~\vspace{8cm}
\newpage
\thispagestyle{empty}

\section{Introduction}
\section{From microservices to automated orchestration}
\subsection{Kubernetes as a Docker containers orchestration system}
\textit{This section describes what Kubernetes is and what problems it solves. Furthermore, the section acknowledges Kubernetes popularity.}

\subsection{Kubernetes architecture}
% this short summary before each section helps me to stay focused on what i want to write about
\textit{This section contains a deep dive into the Kubernetes architecture. The focus is on technical details and describing the responsibilities of each Kubernetes component.}
A \textbf{Kubernetes cluster} is a single unit of computers that are connected to work together. Any such cluster includes two kinds of instances: Masters and Nodes \cite{k8s-cluster}. This is depicted on the figure below:


\printbibliography

\end{document}
