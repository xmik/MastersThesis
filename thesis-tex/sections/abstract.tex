\begin{abstract}
{\scriptsize The presented master’s thesis aims to \textbf{deploy Kubernetes cluster on AWS cloud, in a production environment, using two deployment methods}. The production environment is defined as an environment which satisfies selected requirements. The nine selected requirements included central logging, high availability, and automation.}

{\scriptsize Two deployment methods were chosen: using \textit{kops} and using  \textit{eksctl}. The methods were first theoretically researched and described. Then, they were used in the empirical part to deploy Kubernetes clusters. The methods were a~great help and they facilitated the clusters deployment.}

{\scriptsize \textbf{These two methods of Kubernetes cluster deployment were compared}. Several comparison criteria included time, cost, and the verdict whether a method satisfied all the production environment requirements. The conclusion was that each method can be used for different use cases. \textit{Kops} allows more configuration, whereas  \textit{eksctl} is easier to configure. \textit{Kops} method was deemed faster (in relation to cluster operations) and cheaper.}

{\scriptsize \textbf{Keywords:} \textit{Kubernetes}, \textit{Amazon Web Services (AWS)}, \textit{Amazon Elastic Kubernetes Service (EKS)}, \textit{eksctl}, \textit{kops}, \textit{deployment}, \textit{operations automation}.}

\\
\begin{center}
\textbf{Streszczenie}
\end{center}

{\scriptsize Celem niniejszej pracy dyplomowej było \textbf{przeanalizowanie możliwości wykorzystania klastra Kubernetes w procesie produkcji oprogramowania w aspekcie instalacji i konfiguracji klastra na chmurze AWS na przykładzie dwóch wybranych metod}. Środowisko produkcyjne zdefiniowano jako takie środowisko, które spełnia wybrane wymagania produkcyjne – wyselekcjonowano dziewięć wymagań, m. in.: centralny system logowania, High Availability oraz automatyzacja operacji.}

{\scriptsize Wybrano dwie metody wdrażana klastra Kubernetes: program \textit{kops} oraz program  \textit{eksctl}. Metody te zostały najpierw opisane teoretycznie, a następnie użyto ich w empirycznej części pracy. Obie metody stanowiły wielkie wsparcie w procesie instalowania i konfiguracji klastra.}

{\scriptsize Następnie, \textbf{obie wybrane metody wdrażana klastra Kubernetes porównano}, wykorzystując wybrane kryteria. Dotyczyły one m. in. czasu, kosztu, sprawdzenia, czy każda z metod spełnia wybrane wymogi środowiska produkcyjnego. Po tym badaniu wysnuto wniosek, że każda z wybranych metod może być użyta do wdrożenia klastra Kubernetes w zależności od konkretnych potrzeb deweloperów. Na przykład posługując się programem \textit{kops}, dostępnych jest więcej opcji konfigurujących klaster, a używając programu  \textit{eksctl} łatwiejsza staje się konfiguracja klastra. Metoda używająca \textit{kops} okazała się szybsza (jeśli chodzi o czas wykonywania operacji typu: tworzenie, przetestowanie, usuniecie klastra) i tańsza.}

{\scriptsize \textbf{Słowa kluczowe:} \textit{klaster Kubernetes}, \textit{chmura Amazon Web Services (AWS)}, \textit{Amazon Elastic Kubernetes Service}, \textit{kops}, \textit{eksctl}.}

\end{abstract}
