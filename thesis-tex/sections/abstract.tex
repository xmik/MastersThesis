\begin{abstract}
This study aims to deploy a Kubernetes cluster on AWS cloud, in a production environment, using different deployment methods. The production environment is defined as such an environment which satisfies selected requirements. 9 requirements were selected, including e.g.: central logging, high availability and automation.

Two deployment methods were chosen: using kops and using eksctl. The methods are first theoretically research and described. Then, they are used in the empirical part to deploy Kubernetes clusters. The methods are a great help and facilitate the cluster deployment.

The two methods were compared using several comparison criteria. The criteria contained, among others: time, cost and the verdict whether a deployment method can satisfy all the production environment requirements.  The conclusion was that each method can be used for different use cases. For example: kops allows more configuration, whereas eksctl is easier to configure. The method which was deemed faster (in relation to cluster operations) and cheaper was: kops.

\\
\textbf{Keywords:} Kubernetes, AWS, EKS, eksctl, kops, deployment, automation.
\end{abstract}
