\begin{abstract}
{\scriptsize This study aims to deploy a Kubernetes cluster on AWS cloud, in a production environment, using different deployment methods. The production environment is defined as such an environment which satisfies selected requirements. 9 requirements were selected, including e.g.: central logging, high availability and automation.}

{\scriptsize Two deployment methods were chosen: using kops and using eksctl. The methods are first theoretically research and described. Then, they are used in the empirical part to deploy Kubernetes clusters. The methods are a great help and facilitate the cluster deployment.}

{\scriptsize \textbf{The two methods were compared} using several comparison criteria. The criteria contained, among others: time, cost and the verdict whether a deployment method can satisfy all the production environment requirements.  The conclusion was that each method can be used for different use cases. For example: kops allows more configuration, whereas eksctl is easier to configure. The method which was deemed faster (in relation to cluster operations) and cheaper was: kops.}

\\
\begin{center}
\textbf{Streszczenie}
\end{center}

{\scriptsize Celem tej pracy jest zainstalowanie i skonfigurowanie klustra Kubernetes na chmurze AWS, w środowisku produkcyjnym, wykorzystując różne metody. Środowisko produkcyjne zdefiniowano jako takie środowisko, które spełnia wybrane wymagania. Wyselekcjonowano 9 wymagań, m. in.: centralny system logowania, High Availability oraz automatyzację operacji.}

{\scriptsize Wybrano dwie metody wdrażana klustra Kubernetes: używając programu kops oraz programu eksctl. Metody te zostały najpierw poznane i opisane teoretycznie. Następnie, wykorzystano je w empirycznej części pracy. Obie metody stanowiły wielką pomoc w instalowaniu i konfiguracji klustra.}

{\scriptsize Następnie, \textbf{te dwie wybrane metody porównano}, wykorzystując parę kryteriów. Kryteria te dotyczyły m. in.: czasu, kosztu, sprawdzenia, czy każda z metod spełnia wybrane wymogi środowiska produkcyjnego. Po tym badaniu wysnuto wniosek, że każda z wybranych metod może być użyta w zależności od potrzeb. Na przykład: posługując się programem kops, dostępne jest więcej opcji konfigurujących kluster, a natomiast używając eksctl - konfiguracja jest łatwiejsza. Metoda która okazała się szybsza (jeśli chodzi o czas wykonywania operacji typu: stworzenie, przetestowanie, usunięcie klustra) to: kops.}


\\~\\
{\scriptsize \textbf{Keywords:} Kubernetes, AWS, EKS, eksctl, kops, deployment, automation.}
\end{abstract}
