% 4
\textit{This is a practical section. It includes planning and designing the production deployment, considering: capacity planning, choosing which requirements to satisfy and taking any other deployment and infrastructure related decisions.}
~\\

\subsection{Chosen requirements of production deployment}
\textbf{There are numerous requirements for a production deployment of a Kubernetes cluster}. Some of them were gathered throughout available literature and presented in the section: \ref{Production deployment requirements}. It is common knowledge that companies, which deploy Kubernetes and similar systems, obey some set of best practices, dedicated to these companies only. Thus, the requirements presented in this work do not exhaust the topic.

MAYBE TODO:
Furthermore, the author if this work decided to satisfy only a few of the production deployment requirements. The reason - because i don't want to.

\begin{itemize}
\item Passing tests, a healthy cluster
\item Automation and Infrastructure as Code
\item Central Monitoring
\item Central Logging
\item Backup
\item HA - maybe
\item Autoscaling - maybe
\item Security - maybe
\item Live Cluster Upgrades - rather not
\item Audit - rather not
\end{itemize}

\subsubsection{Designing automated tests}

\subsection{Capacity planning}
Virtualization and chicken-counting (0, 1, many) are your friends here. Virtu-
alization makes it easy to create an environment that represents the important
aspects of your production environment, while being able to run on a single
physical machine. Chicken-counting means that if your production site has
250 web servers, 2 should be enough to represent the significant process
boundaries. book-cicd, p. 254

\subsection{Other decisions and configuration}

\subsection{Expected cost}



troubleshooting k8s -mastering k8s p. 58
