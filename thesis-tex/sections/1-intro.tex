\section{Introduction}

% this should take 2 or 3 pages;
% the subsections are not mandatory

\subsection{Topic and problem description}

Microservices, Continuous Integration and Delivery, Docker, DevOps, Infrastructure as Code - these are the current trends and buzzwords in the technological world of 2020. A popular tool which can \textbf{facilitate the deployment and maintenance of microservices is Kubernetes}. Kubernetes is a platform for running containerized applications, for example: microservices. The first part of the problem, which this thesis is trying to solve is: \textbf{how to deploy Kubernetes itself}. The second part is: how to ensure that the deployment fulfills the needs of a production environment.

Any application may comprise several components, for example: a backend server, a frontend server, database. It is common knowledge, that deploying an application to a production environment, should obey a set of guidelines. It should be easy to view all the log messages generated by each of the components of the application, the application should be reachable for its end users and also, it would be nice if in a case of any component failure - that component should be available despite the failure. \textbf{Kubernetes facilitates satisfying such requirements. But, the aim of this thesis is to ensure such requirements for Kubernetes itself}. A set of requirements will be selected. Then, several ideas will be provided on how to meet each of the chosen requirements.

\textbf{There are plenty methods of Kubernetes cluster deployment. Many will be described}. The methods differ in relation to: how much customization they offer, which clouds they support, how much they cost. Some of the methods has existed since the Kubernetes was created, the other ones, like \textit{AWS EKS}, has been invented later. The latter methods are harder to find in books tackling the Kubernetes deployment task, but there are other sources which explain how to use them (mostly: official documentation of the methods and blog posts).

There exist sources which compare several methods of Kubernetes cluster deployment. But, they are either non-formal sources (e.g. blog posts or Internet tutorials) or they do not compare the two methods selected in this thesis or they do not consider the production environment. Therefore, \textbf{this thesis has the opportunity to offer some novelty}.

\subsection{Aim and scope of this study}


Nowadays, the world is full of choices. The technology is exuberant. However, time, as our resource, is limited. Thus, it is often advisable to use an already existing solution instead of inventing our own. It is even better if there are many such solutions. The presented Master's thesis aims \textbf{to compare two methods of deploying a Kubernetes cluster}. Both of the methods concern AWS cloud. AWS cloud was chosen mainly because of its wide popularity and the range of provided services. Besides the two chosen methods of deployment, there are many more, including the DIY method and deploying on-premises.

This thesis attempts \textbf{to stipulate the requirements of a production environment}. Then, the requirements are used as comparison criteria to help assess the two methods of deployment. It is expected that one method could be easier to use than the other but also a method could be insufficient to satisfy all the production environment requirements. Furthermore, other criteria will be used, such as: cost of both methods and amount of problems encountered.

It is intended, that this work should \textbf{focus on the practical aspect of a Kubernetes cluster deployment}. The limitations and known issues of both methods are going to be described. Therefore, the thesis might be helpful to the engineers or consultants who are responsible for Kubernetes cluster deployment.

There are already some literature sources that compare chosen methods of Kubernetes cluster deployment. However, they are not constructed in a scientific, formal form (they are either blog posts or tutorials available in the Internet) or they do not consider a production environment.

\subsection{Structure of this thesis}

The presented thesis is divided into seven chapters.

The first chapter is \textbf{an introduction}. It describes briefly the topic and the problem. Then, it includes aim and scope of this study. And then, the structure of this study is presented.

The second chapter focuses on \textbf{basic information concerning: microservices, DevOps, Docker and Kubernetes}. Furthermore, AWS cloud is described there. The chapter ends with stipulating the requirements of a production deployment.

In the next, third, chapter \textbf{the most popular methods of Kubernetes cluster deployment are described}. The two methods that will be compared in the later part are also included.

The fourth chapter is a practical one. It provides \textbf{planning and designing of the production deployment} which will be conducted later in the thesis. Some important decisions are taken here, for example: which Kubernetes version to use or which tools to use.

Then, the fifth chapter \textbf{refers to code} and it is the core part of this Master's thesis. It provides \textbf{the steps that were used to deploy Kubernetes clusters on AWS, using two methods}:
\begin{enumerate}
\item deployment with \textit{eksctl} on AWS EKS service,
\item deployment with \textit{kops} on AWS EC2 instances.
\end{enumerate}
Anyone following these steps should be capable to recreate the same Kubernetes clusters as described here and thus, it should be possible to draw the same conclusions as the author of this thesis did. Furthermore, all the encountered problems and suggested solutions are provided.

The sixth chapter \textbf{presents the comparison} of the two deployment methods using chosen comparison criteria. Each subsection of this chapter deals with a single comparison criterion.

Finally, the last, seventh, chapter offers \textbf{the summary}, briefly describes the lessons learned and also provides some ideas for future work.
