\section{Deployment of a Kubernetes cluster, using various methods}
\textit{This is a practical chapter and it was written together with performing the empirical work and writing the source code. Main steps of Kubernetes cluster deployment will be enumerated. Also, encountered problems will be listed and potential solutions will be presented.}
\\

\subsection{Experimental deployment using Kops}
\textit{In this subsection first experiment with Kops is described. This was an attempt of deploying a Kubernetes cluster on AWS cloud with no prerequisite work done}
\\

Although the aim of this work is to create a production Kubernetes cluster, it is always welcome, when there is a possibility to start working with a program easily. It is nice to have a simple working proof of concept. Thus, it was decided to start with Kops without performing any prerequisite steps. The following commands were invoked:
\begin{lstlisting}[basicstyle=\small,caption={Commands used to create a cluster with kops, without prerequisite steps performed},captionpos=b,language=Bash,xleftmargin=1cm]
$ export KOPS_STATE_STORE=s3://dummy-k8s-kops-state-store
$ export NAME=dummy-k8s-kops.k8s.local
$ kops create cluster \
--master-zones=eu-west-1a \
--zones=eu-west-1a \
--node-count=1 \
--node-size=t2.nano \
--master-size=t2.nano \
${NAME}
\end{lstlisting}

AWS instance types were chosen (using the CLI flags: \textit{--node-size} and \textit{--master-size}) and also the number of Kubernetes worker nodes was set (using the CLI flag: \textit{--node-count}), so that, if the cluster had been created, its cost would have been under control. \textbf{As expected and what is aligned with the Kops documentation\cite{online-kops-aws}, the above commands failed}, because there was no dummy-k8s-kops-state-store S3 bucket created. The command failed with the following output:
\begin{lstlisting}[basicstyle=\small,caption={Output of the commands used to create a cluster with Kops, without prerequisite steps performed},captionpos=b,language=Bash,xleftmargin=1cm]
error reading cluster configuration "dummy-k8s-kops.k8s.local":
error reading s3://dummy-k8s-kops-state-store/dummy-k8s-kops.k8s.local/config:
Could not retrieve location for AWS bucket dummy-k8s-kops-state-store
\end{lstlisting}

The Kops documentation\cite{online-kops-aws} informs that the following goals should be first accomplished, before deploying a Kubernetes cluster:
\begin{itemize}
\item AWS CLI tools should be installed
\item AWS credentials should be set
\item AWS IAM user and its permissions should be set
\item DNS should be configured
\item An S3 bucket should be created for storing a cluster state
\item AWS Region and Availability Zones should be chosen
\end{itemize}

Further in this chapter, after all the prerequisites will have been met, Kops will be used to create a production Kubernetes cluster.

\subsection{Experimental deployment using eksctl}
\textit{In this subsection first experiment with eksctl is described. This was an attempt of deploying a Kubernetes cluster on AWS cloud with no prerequisite work done}
\\

In order to be consistent, the same first experiment was performed using eksctl. First a YAML configuration file was created and then the eksctl CLI command was invoked:
\begin{lstlisting}[basicstyle=\small,caption={Commands used to create a cluster with eksctl, without prerequisite steps performed},captionpos=b,language=Bash,xleftmargin=1cm]
$ cat cluster.yaml
apiVersion: eksctl.io/v1alpha5
kind: ClusterConfig

metadata:
  name: cluster-eks
  region: eu-west-1

nodeGroups:
  - name: ng-1
    labels: { role: worker, cluster: cluster-eks }
    instanceType: t2.nano
    desiredCapacity: 1
    ssh:
      allow: true
$ eksctl create cluster -f cluster.yaml
\end{lstlisting}

These commands resulted in a successful creation of a cluster in "eu-west-1" AWS region with one worker node. Apart from that, the configuration file needed to access the remote cluster (remote, because deployed on AWS) was automatically created and written to: \textit{~/.kube/config}. In order to verify that the worker nodes were running, the following command was run:
\begin{lstlisting}[basicstyle=\small,caption={Command used to list Kubernetes worker nodes to verify that one such node was running},captionpos=b,language=Bash,xleftmargin=1cm]
$ kubectl get nodes
NAME                                           STATUS   ROLES    AGE    VERSION
ip-192-168-176-90.eu-west-1.compute.internal   Ready    <none>   3m8s   v1.16.8
\end{lstlisting}

This experiment was successful. \textbf{It was easy to deploy a Kubernetes cluster using eksctl. No prerequisite steps were needed}. Besides, it was also easy to set up the YAML configuration file, basing on the eksctl documentation\cite{eksctl-creating-clusters}.


The cluster was then deleted with the following command:
\begin{lstlisting}[basicstyle=\small,caption={Command used to delete Kubernetes cluster with eksctl},captionpos=b,language=Bash,xleftmargin=1cm]
$ eksctl delete cluster -f cluster.yaml --wait
\end{lstlisting}
The \textit{--wait} CLI flag was applied. Without it, a delete operation would have been only requested but not waited for. In some cases it happens that the deletion fails, and, without this flag, the errors would not have been propagated back as the CLI command output. Then, one would be forced to delete the AWS resources manually\cite{eksctl-creating-clusters}.

\subsection{Production deployment using Kops}
\textit{This section briefly presents all the steps performed that lead to a Kubernetes cluster deployment on the AWS cloud using Kops. Here an attempt was made to satisfy all the production environment requirements selected in the chapter: \ref{prep-prod}.}
\\

\subsubsection{First working cluster}

First, all the prerequisites were done. In order to make things simpler, an AWS user with administrator permissions was used. SSH keypair was already set. Then, it was decided to use the gossip-based DNS. Then, an S3 bucket was created to keep the Kops cluster configuration. Both versioning and server side encryption of the S3 bucket were enabled. Versioning was strongly recommended, because thanks to it, one may revert or recover a previous cluster state store. S3 bucket encryption is not required, but may be needed for compliance reasons\cite{online-kops-aws}. Setting the S3 bucket was done by the following commands:
\begin{lstlisting}[basicstyle=\tiny,caption={Commands used to set an AWS S3 bucket for Kops},captionpos=b,language=Bash,xleftmargin=1cm]
$ export K8S_EXP_REGION="eu-west-1"
$ export K8S_EXP_KOPS_S3_BUCKET="k8s-kops-for-masters-thesis.k8s.local"
$ export K8S_EXP_ENVIRONMENT="testing"
$ export K8S_EXP_CLUSTER_NAME="${K8S_EXP_ENVIRONMENT}.${K8S_EXP_KOPS_S3_BUCKET}"
$ aws s3api create-bucket --bucket ${K8S_EXP_KOPS_S3_BUCKET} --region ${K8S_EXP_REGION} \
--create-bucket-configuration LocationConstraint=${K8S_EXP_REGION}
$ aws s3api put-bucket-versioning --bucket ${K8S_EXP_KOPS_S3_BUCKET} \
--versioning-configuration Status=Enabled
$ aws s3api put-bucket-encryption --bucket ${K8S_EXP_KOPS_S3_BUCKET} \
--server-side-encryption-configuration '{"Rules":[{"ApplyServerSideEncryptionByDefault":{"SSEAlgorithm":"AES256"}}]}'
\end{lstlisting}

Then, this command was used to make Kops create the cluster configuration and put it in the S3 bucket:
\begin{lstlisting}[basicstyle=\small,caption={Command used to make Kops create the cluster configuration and put it in the S3 bucket},captionpos=b,language=Bash,xleftmargin=1cm]
$ kops create cluster --state "s3://${K8S_EXP_KOPS_S3_BUCKET}" \
--master-zones=eu-west-1a --master-count=1 --master-size=t2.nano \
--zones=eu-west-1a --node-count=1 --node-size=t2.nano \
${K8S_EXP_CLUSTER_NAME}
\end{lstlisting}

It resulted in a not successful exit status (1) and returned the following output:
\begin{lstlisting}[basicstyle=\tiny,caption={Output from command used to make Kops create the cluster configuration and put it in the S3 bucket},captionpos=b,language=Bash,xleftmargin=1cm]
error building tasks: error remapping manifest addons/kops-controller.addons.k8s.io/k8s-1.16.yaml: \
error parsing yaml: error converting YAML to JSON: yaml: line 56: did not find expected alphabetic or numeric character
\end{lstlisting}
The reason for this was that the development environment had a non-numeric environment variable set. It was a password and the variable value contained asterisks (\textit{****}). After the variable was unset (with the bash command: \textit{unset}), the \textit{kops create cluster} succeeded. The output of this command presented the list of actions which Kops will perform on the AWS account, e.g. creating EBS volumes for Etcd, configuring IAM, creating keypairs for Kubernetes services, configuring network and setting EC2 instances. Details of the to-be-created resources were also provided, for example, the command output informed that the EC2 image will be: \textit{kope.io/k8s-1.16-debian-stretch-amd64-hvm-ebs-2020-01-17}. Apart from printing the output, Kops created a directory named the same as the cluster name (\textit{testing.k8s-kops-for-masters-thesis.k8s.local}) in the S3 bucket. Among the files automatically created by Kops, there is a configuration file named: \textit{config} and it contains the cluster settings. It can be changed from command line with the following command:
\begin{lstlisting}[basicstyle=\small,caption={Command used to edit a Kubernetes cluster managed by Kops},captionpos=b,language=Bash,xleftmargin=1cm]
$ kops edit cluster ${K8S_EXP_CLUSTER_NAME} \
--state "s3://${K8S_EXP_KOPS_S3_BUCKET}"
\end{lstlisting}
Running this command starts a vim session. After editing the configuration, the cluster can be created or updated (if it was created earlier) with the following command:
\begin{lstlisting}[basicstyle=\small,caption={Command used to edit a Kubernetes cluster managed by Kops},captionpos=b,language=Bash,xleftmargin=1cm]
$ kops update cluster ${K8S_EXP_CLUSTER_NAME} \
--state "s3://${K8S_EXP_KOPS_S3_BUCKET}" --yes
\end{lstlisting}

The above command deploys a cluster on AWS and prints a helpful output, which ends with:
\begin{lstlisting}[basicstyle=\small,caption={Output from the command: kops update cluster},captionpos=b,language=Bash,xleftmargin=1cm]
Cluster is starting.  It should be ready in a few minutes.

Suggestions:
 * validate cluster: kops validate cluster
 * list nodes: kubectl get nodes --show-labels
 * ssh to the master: ssh -i ~/.ssh/id_rsa admin@api.testing.k8s-kops-for-masters-thesis.k8s.local
 * the admin user is specific to Debian. If not using Debian please use the appropriate user based on your OS.
 * read about installing addons at: https://github.com/kubernetes/kops/blob/master/docs/operations/addons.md.
\end{lstlisting}

Unfortunately, the commands listed as suggestions above did not work. They resulted in:
\begin{lstlisting}[basicstyle=\tiny,caption={Running commands to connect with a cluster created by Kops together with returned output},captionpos=b,language=Bash,xleftmargin=1cm]
$ kops validate cluster ${K8S_EXP_CLUSTER_NAME} --state "s3://${K8S_EXP_KOPS_S3_BUCKET}"
Validating cluster testing.k8s-kops-for-masters-thesis.k8s.local
unexpected error during validation: error listing nodes: \
Get https://api-testing-k8s-kops-for--l9puut-394396927.eu-west-1.elb.amazonaws.com/api/v1/nodes: EOF

$ ssh -i \
"kubernetes.testing.k8s-kops-for-masters-thesis.k8s.local-62:32:29:2c:81:d1:4f:85:f7:0a:57:3c:db:22:bd:85.pem" \
root@ec2-3-250-94-31.eu-west-1.compute.amazonaws.com
Warning: Identity file \
kubernetes.testing.k8s-kops-for-masters-thesis.k8s.local-62:32:29:2c:81:d1:4f:85:f7:0a:57:3c:db:22:bd:85.pem \
not accessible: No such file or directory.
^C

$ ssh -i ~/.ssh/id_rsa admin@api.testing.k8s-kops-for-masters-thesis.k8s.local
ssh: Could not resolve hostname api.testing.k8s-kops-for-masters-thesis.k8s.local: \
Name does not resolve
\end{lstlisting}

There was however no problem in deleting the cluster. It took several minutes, but this command succeeded and all the AWS resources (except for the manually created S3 bucket) were deleted:

\begin{lstlisting}[basicstyle=\small,caption={Command used to delete a Kubernetes cluster created with Kops},captionpos=b,language=Bash,xleftmargin=1cm]
$ kops delete cluster --name ${K8S_EXP_CLUSTER_NAME} --state "s3://${K8S_EXP_KOPS_S3_BUCKET}" --yes
\end{lstlisting}


\subsubsection{Cluster which satisfies all the production deployment requirements TODO}


labels https://github.com/kubernetes/kops/blob/master/docs/manifests_and_customizing_via_api.md https://github.com/kubernetes/kops/blob/master/docs/labels.md
ci yaml config

\subsection{Production deployment using eksctl}
\textit{This section briefly presents all the steps performed that lead to a Kubernetes cluster deployment on the AWS cloud using eksctl. Here an attempt was made to satisfy all the production environment requirements selected in the chapter: \ref{prep-prod}.}
\\

\subsection{TODO}

troubleshooting k8s -mastering k8s p. 58
* what happens if we manually delete a iptables rule? kube-proxy will put it back after 10 to 30s - https://learnk8s.io/blog/kubernetes-chaos-engineering-lessons-learned
* https://learnk8s.io/troubleshooting-deployments - must read!!
