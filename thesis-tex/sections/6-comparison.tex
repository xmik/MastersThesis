\section{Comparison of the used methods}
\textit{In this chapter, several comparison criteria will be used in order to compare the two chosen methods of a Kubernetes cluster deployment. The chapter will end with a summary deciding which method was better in relation to which comparison criterion.}

\subsection{Additional steps needed to create a Kubernetes cluster}
* kubeconfig automatically edited by both kops and eksctl
* prerequisites; eksctl - none; kops - s3 bucket;
* command creating the cluster waited for the cluster to be finished: kops - not; eksctl - ??
* can api server flags be changed?
* kops default img: debian; eksctl - amazon linux 2

\subsection{Cost}

* kops, s3 bucket for cluster store, versioning and encryption enabled,  The objects are encrypted using server-side encryption with either Amazon S3-managed keys (SSE-S3) or customer master keys (CMKs) stored in AWS Key Management Service (AWS KMS). https://docs.aws.amazon.com/AmazonS3/latest/dev/bucket-encryption.html; "There are no new charges for using default encryption for S3 buckets. Requests to configure the default encryption feature incur standard Amazon S3 request charges. " AES-256 (SSE-S3) https://docs.amazonaws.cn/en_us/AmazonS3/latest/user-guide/default-bucket-encryption.html; s3 data transfer pricing; maybe storage classes
* ec2 instances
* load balancer for cluster as api endpoint and for each ing resources (kops) (avoiding LB: consider NodePort instead or also - would it work with curl http://api-endpoint -H 'my-ingress-url'?)
* vpc costs?

\subsection{Problems count}
* OS_PASSWORD variable problem with kops
* t2.nano instance types not working with kops

\subsection{Production requirements which could not be satisfied}
\subsection{Resources which could not be automated}
\subsection{Summary}
